\documentclass[12pt]{article}
\usepackage{amsmath}
\usepackage{amsthm}
\usepackage{txfonts}
\usepackage[margin=1.25in]{geometry}

\newcommand{\iso}{\cong}
\newcommand{\siso}{\ensuremath{\iso_s}}
\newcommand{\fiso}{\iso_f}
\DeclareMathOperator{\Aut}{Aut}
\newcommand{\Auts}{\Aut_s}

\newtheorem{prop}{Proposition}
\newtheorem{conj}[prop]{Conjecture}

\begin{document}

\begin{center}
    Notes on Semi-Isomorphism
\end{center}

Let $G$ and $H$ be simple graphs with adjacency matrices $A$ and $B$,
respectively. We say that these graphs are \emph{semi-isomorphic}
provided there are permutation matrices $P$ and $Q$ such that
$AP=QB$. Notation $G \siso H$.

We also recall that $G$ and $H$ are \emph{fractionally isomorphic} if
there is a doubly stochastic matrix $S$ such that $AS=SB$. Notation:
$G\fiso H$.

\begin{prop}
  Semi-isomorphism is an equivalence relation.
\end{prop}

\begin{proof}
  It is easy to see that \siso\ is reflexive: $AI=IA$.

  Suppose $G \siso H$. Let $A$ and $B$ be the adjacency matrices, and
  let $P$ and $Q$ be permutation matrices with $AP=QB$. Taking the
  transpose of both sides gives
  \[
  (AP)'=(QB)' \Rightarrow P'A=BQ' \Rightarrow BQ' = P'A
  \]
  and so $H\siso G$. Therefore \siso\ is symmetric.

  Finally, suppose $G$, $H$, and $K$ are graphs with $G\siso H\siso
  K$. Let $A$, $B$, and $C$ be their adjacency matrices. We therefore
  have permutation matrices $P$, $Q$, $R$, and $S$ such that
  \[
  AP=QB \quad\text{and}\quad BR = SC .
  \]
  The first gives $Q'AP = B$ and we substitute into the second to give
  $[Q'AP]R=SC$. Multiplying through by $Q$ gives $A[PR]= [QS]C$, and
  so $G\siso K$. Therefore \siso\ is transitive.
\end{proof}


\begin{prop}
  Let $n\ge3$ be an integer. Then $C_n+C_n \siso C_{2n}$ if and only
  if $n$ is odd.
\end{prop}

\begin{proof}
  Let $A_0$ be the adjacency matrix of $C_n$. For example,
  for $n=7$ we have
  \[
  A_0 = 
  \left[
    \begin{array}{ccccccc}
      0 & 1 & 0 & 0 & 0 & 0 & 1 \\
      1 & 0 & 1 & 0 & 0 & 0 & 0 \\
      0 & 1 & 0 & 1 & 0 & 0 & 0 \\
      0 & 0 & 1 & 0 & 1 & 0 & 0 \\
      0 & 0 & 0 & 1 & 0 & 1 & 0 \\
      0 & 0 & 0 & 0 & 1 & 0 & 1 \\
      1 & 0 & 0 & 0 & 0 & 1 & 0 \\
    \end{array}
  \right] .
  \]
  Indexing vertices from $0$, vertex $k$ is adjacent to $k\pm1 \pmod2$.

  Observe that
  \[
  A =
  \begin{bmatrix}
    A_0 & 0 \\ 0 & A_0
  \end{bmatrix}
  \]
  is the adjacency matrix of $C_n+C_n$. 

  Now let 
  \[
  B = 
  \begin{bmatrix}
    0 & A_0 \\ A_0 & 0
  \end{bmatrix}
  \]
  and let $H$ be the graph whose adjacency matrix is $B$.

  Computational evidence only so far. This should be doable.
\end{proof}


\begin{prop}
  Let $G$ and $H$ be graphs. Then
  \[
  G\iso H \Rightarrow    G\siso H  \Rightarrow G\fiso H  
  \]
  but neither reverse implication holds. 
\end{prop}

\begin{proof}
  The first implication is trivial. 
    
  Suppose $G\siso H$. Let $A$ and $B$ be their adjacency matrices and
  let $P$ and $Q$ be permutation matrices with $AP=QB$. Transpose both
  sides to give $(AP)' = (QB)' \Rightarrow P'A = BQ' \Rightarrow
  P(P'A)Q = P(BQ')Q \Rightarrow AQ=PB$.
    
  Add the equations $AP=QB$ and $AQ=PB$ to give $A(P+Q)=(P+Q)B$. Note
  that $S=\frac12(P+Q)$ is doubly stochastic and that
  $AS=SB$. Therefore $G \fiso H$.

  Calculations show
  \[
  \begin{aligned}
    C_{10} &\siso C_5+C_5 \qquad\text{but}\qquad C_{10}\not\iso C_5+C_5 \\
    C_{10} &\fiso C_4+C_6 \qquad\text{but}\qquad C_{10}\not\siso C_4+C_6
  \end{aligned}
  \]
  and so neither reverse implication holds.
\end{proof}






\begin{conj}
  Let $G$ and $H$ be graphs with adjacency matrices $A$ and $B$,
  respectively. Suppose that $G \fiso H$ with $AS=SB$ where $S$ is a
  doubly stochastic matrix whose entries consist only of the values
  $0$, $\frac12$, and $1$. Then $G\siso H$. 
\end{conj}

\noindent\textbf{This is false!} Counterexample: $2C_4$ and $C_8$ are not
semi-isomorphic but they are fractionally isomorphic. Let $A$ and $B$
be the adjacency matrices of these two graphs:
\[
A = 
\left[
\begin{array}{cccccccc}
0 & 1 & 0 & 1 & 0 & 0 & 0 & 0 \\
1 & 0 & 1 & 0 & 0 & 0 & 0 & 0 \\
0 & 1 & 0 & 1 & 0 & 0 & 0 & 0 \\
1 & 0 & 1 & 0 & 0 & 0 & 0 & 0 \\
0 & 0 & 0 & 0 & 0 & 1 & 0 & 1 \\
0 & 0 & 0 & 0 & 1 & 0 & 1 & 0 \\
0 & 0 & 0 & 0 & 0 & 1 & 0 & 1 \\
0 & 0 & 0 & 0 & 1 & 0 & 1 & 0 \\
\end{array}
\right]
\quad\text{and}\quad
B = 
\left[
\begin{array}{cccccccc}
0 & 1 & 0 & 0 & 0 & 0 & 0 & 1 \\
1 & 0 & 1 & 0 & 0 & 0 & 0 & 0 \\
0 & 1 & 0 & 1 & 0 & 0 & 0 & 0 \\
0 & 0 & 1 & 0 & 1 & 0 & 0 & 0 \\
0 & 0 & 0 & 1 & 0 & 1 & 0 & 0 \\
0 & 0 & 0 & 0 & 1 & 0 & 1 & 0 \\
0 & 0 & 0 & 0 & 0 & 1 & 0 & 1 \\
1 & 0 & 0 & 0 & 0 & 0 & 1 & 0 \\
\end{array}
\right]
\]
and let
\[
S = \frac12\left[
\begin{array}{cccccccc}
0 & 0 & 0 & 1 & 0 & 0 & 0 & 1 \\
0 & 0 & 1 & 0 & 0 & 0 & 1 & 0 \\
0 & 1 & 0 & 0 & 0 & 1 & 0 & 0 \\
1 & 0 & 0 & 0 & 1 & 0 & 0 & 0 \\
0 & 0 & 1 & 0 & 0 & 0 & 1 & 0 \\
0 & 0 & 0 & 1 & 0 & 0 & 0 & 1 \\
1 & 0 & 0 & 0 & 1 & 0 & 0 & 0 \\
0 & 1 & 0 & 0 & 0 & 1 & 0 & 0 \\
\end{array}
\right].
\]
We have $AS=SB$.




\vskip 0.5in

Let $G$ be a graph with adjacency matrix $A$. A
\emph{semi-automorphism} of $G$ is a pair of permutation matrices
$(P,Q)$ such that $AP=QA$. The set of all semi-automorphisms of $G$ is
$\Auts(G)$.

\begin{prop}
  If $(P,Q)\in\Auts(G)$, then $(Q',P')\in\Auts(G)$.
\end{prop}
\begin{proof}
  $(P,Q)\in\Auts(G) \Rightarrow AP=QA \Rightarrow (AP)'=(QA)'
  \Rightarrow P'A = A Q' \Rightarrow AQ'=P'A \Rightarrow (
  Q',P')\in\Auts(G)$.
\end{proof}


\begin{prop}
  If $(P,Q) \in \Auts(G)$, then $(P',Q') \in \Auts(G)$.
\end{prop}

\begin{proof}
  $(P,Q) \in \Auts(G) \Rightarrow AP=QA \Rightarrow Q'AP=A \Rightarrow
  Q'A=AP' \Rightarrow AP'=Q'A \Rightarrow (P',Q') \in \Auts(G).$
\end{proof}

Combining these two results gives this:
\begin{prop}
  If $(P,Q) \in \Auts(G)$, then $(Q,P) \in \Auts(G)$.\qed
\end{prop}


\begin{prop}
  If $(P,Q), (X,Y) \in \Auts(G)$, then $(PX,QY)\in\Auts(G)$.
\end{prop}

\begin{proof}
  Since $(X,Y)\in\Auts(G)$, we also have $(X',Y')\in\Auts(G)$. This
  gives
  \[
  AP=QA \quad\text{and}\quad AX'=Y'A 
  \] 
  and so 
  \[
  Q'AP = A = YAX'.
  \]
  Left multiply by $Q$ and right multiply by $X$ to give 
  \[
  Q[Q'AP]X = Q[YAX']X \Rightarrow A(PX) = (QY)A     
  \]
  and so $(PX,QY)\in\Auts(G)$.
\end{proof}

We now define an operation for $\Auts(G)$. For $(P,Q), (X,Y) \in
\Auts(G)$, let
\[
(P,Q) * (X,Y) = (PQ,XY).    
\]

Note that $(I,I)$ is an identity element for this operation and the
inverse of $(P,Q)$ is $(P',Q')$. Therefore $\Auts(G)$, together with
the operation $*$, is a group that we call the \emph{semi-automorphism
  group} of the graph $G$.

Note that $P$ is an automorphism of $G$ iff $(P,P) \in \Auts(G)$. This
gives the following:
\begin{prop}
  The automorphism group $\Aut(G)$ of a graph $G$ is (isomorphic to) a
  subgroup of its semi-automorphism group $\Auts(G)$.\qed
\end{prop}

\begin{prop}
  Let $G,H_1,H_2$ be graphs with $G\siso H_1$ and $H_1 \iso H_2$. Then
  $G\siso H_2$. 
\end{prop}

\begin{proof}
  Let $A,B_1,B_2$ be the adjacency matrices of $G, H_1, H_2$,
  respectively. Then we have
  \[
  AP = QB_1 \qquad\text{and}\qquad R' B_1 R = B_2
  \]
  where $P,Q,R$ are permutation matrices. Rewrite the second equation
  as $B_1= R B_2 R'$ and substitute in the first to give
  \[
  AP = Q[R B_2 R'] \Rightarrow A(PR)= (QR) B_2
  \]
  and so $G \siso H_2$.
\end{proof}


Given a graph $G$ with adjacency matrix $A$, we can find all graphs
semi-isomorphic to $G$ by considering all permutation matrices $P$ and
$Q$ for which $Q'AP$ is a legitimate adjacency matrix. If $G$ has $n$
vertices, there are $(n!)^2$ possibilities to consider. However, it is
sufficient to just consider matrices of the form $AP$ (there are
``only'' $n!$ to consider) because if $B=AP$ is an adjacency matrix of
a graph $H$, then $AP = IB$ and so $G\siso H$. Conversely, if $G \siso
H$ with $AP=QB$, then $A(PQ') = QBQ'$. Note that $QBQ'$ is the
adjacency matrix of a graph isomorphic to $H$. Thus listing all
adjacency matrices of the form $AP$ will identify all graphs (up to
isomorphism) that are semi-isomorphic to $G$.


\end{document}